\documentclass[10pt,color=usenames,dvipsnames]{beamer}

\usepackage{graphicx}
\usepackage{listings}

\mode<presentation> {

%\usetheme{Madrid}
\usetheme{Boadilla}

\usepackage{qrcode}
\usepackage{multirow}
\usepackage[utf8]{inputenc}

\usepackage{tikz}
\usetikzlibrary{shapes.geometric, arrows}

\usepackage[export]{adjustbox}

\definecolor{bokugreen}{rgb}{0, 0.49, 0}

%\setbeamercolor{palette primary}{fg=bokugreen}
%\setbeamercolor{palette secondary}{bg=white,fg=bokugreen}
%\setbeamercolor{palette tertiary}{bg=white,fg=bokugreen}
%\setbeamercolor{palette quaternary}{bg=white,fg=bokugreen}
\setbeamercolor{structure}{fg=bokugreen} % itemize, enumerate, etc
%\setbeamercolor{section in toc}{fg=bokugreen,bg=white} % TOC sections
\setbeamertemplate{section in toc}[sections numbered]
\setbeamertemplate{subsection in toc}[subsections numbered]


%\usebackgroundtemplate%
%{%
%	\vspace{-0.5cm}
%	\hspace{9.5cm}
%	\includegraphics{bokulogo.png}%
%
%}

\usepackage{hyperref}
\hypersetup{%
    % warning: color makes also QR code colored, therefore commented out
	colorlinks=true,
    allcolors=bokugreen
    % Uargh! Underlining hyperrefs makes troubles with beamer's pause... :-/
    % https://tex.stackexchange.com/q/262651/8964
}

%gets rid of bottom navigation bars
\setbeamertemplate{footline}[frame number]

%gets rid of bottom navigation symbols
\setbeamertemplate{navigation symbols}{}

% Override palette coloring with secondary
%\setbeamercolor{subsection in head/foot}{bg=UBCgrey,fg=white}


%\usecolortheme{lily}
\useoutertheme{infolines}

}

\usepackage{booktabs}
\usepackage{tikz}

% Thin fonts
\usepackage{cmbright}
\usepackage[T1]{fontenc}

% add git commit hash
\usepackage{xstring}
\usepackage{catchfile}
\CatchFileDef{\HEAD}{../.git/refs/heads/master}{}
\newcommand{\gitrevision}{%
  \StrLeft{\HEAD}{7}%
}


%\definecolor{dark_grey}{gray}{0.5}
%\setbeamercolor{normal text}{fg=dark_grey,bg=white}
%\setbeamertemplate{navigation symbols}{}

%\setbeamercolor*{palette primary}{fg=gray!100,bg=gray!10}
%\setbeamercolor*{palette quaternary}{fg=gray!100,bg=gray!10}
%\setbeamercolor*{palette secondary}{fg=gray!100,bg=gray!20}
%\setbeamercolor*{palette tertiary}{fg=gray!100,bg=gray!10}
%\setbeamercolor*{navigation symbols}{fg=white,bg=white}
\usefonttheme{default}

\setbeamertemplate{blocks}[rounded][shadow=false]
%\setbeamercolor{block title}{bg=gray!10}
%\setbeamercolor{block body}{fg=gray,bg=gray!10}
%\setbeamercolor{frametitle}{fg=}

\setbeamertemplate{frametitle}[default][center]

\setbeamertemplate{itemize items}[default]
\setbeamertemplate{enumerate items}[default]

\newcommand{\F}{\mathbb{F}}

\setbeamertemplate{title page}[default][colsep=-4bp,rounded=true]
\setbeamertemplate{frametitle}[default][left]
%\addtobeamertemplate{frametitle}{}{\vspace{4em}} % increase


\title[Scientific Computing]{Scientific Computing}
\author{Peter Regner, Johannes Schmidt}
\institute{Institute for Sustainable Economic Development, BOKU, Wien}

\subtitle{Command line, GIT and version control}
\date{19.3.2020}
\begin{document}

% Title Page
\begin{frame}[plain]
    \maketitle
    \begin{center}
        \includegraphics[height=1.7cm]{../common/boku-logo.pdf}\\
    \end{center}
    \vfill
    {
        \tiny
        Latest version available on
        \href{https://github.com/inwe-boku/lecture-scientific-computing/}{Github},
        this PDF is version \gitrevision.
    }
\end{frame}


\begin{frame}

	\tableofcontents

\end{frame}

\section{Using a command line}

\begin{frame}[fragile]{Command line: Live Demo}
    Live Demo:
    \begin{verbatim}
    cowsay -f snowman "I'm sweating"
    \end{verbatim}
    \begin{verbatim}
    cowthink "meh."
    \end{verbatim}
\end{frame}

\begin{frame}[fragile]{What is a command line?}
    A command line is a computer interface, which allows you to type commands with parameters
    and to receive text output:\\

    Type something like:
    \begin{verbatim}
        command parameter1 parameter2[ENTER]\end{verbatim}
    Then the \verb|command| will run and output will be printed on the screen.
    \bigskip
    \pause
    \begin{itemize}
        \item Often used as synonyms: terminal, shell, command line, console
        \item Everything about command lines refers to Linux command lines, but git-bash in Windows
            and the terminal in Mac OS are very similar, but the standard Windows terminal is quite different
            (don't use it: PITA).
    \end{itemize}
\end{frame}

\begin{frame}[fragile]{Command line: Syntax}

    \begin{itemize}
        \item Command and all parameters are space-separated
        \item Parameters containing spaces (e.g. file paths) need to be wrapped in double or single
            quotes:
            \begin{verbatim}
    ls "path/to a folder with spaces/subfolder"\end{verbatim}
        \item There are different standard forms of parameters:
        \begin{itemize}
            \item \verb|command parameter|
            \item \verb|command --parameter-name=parameter|
            \item \verb|command --parameter-name parameter|
            \item \verb|command -p parameter|
            \item \verb|command -p=parameter|
        \end{itemize}
    \end{itemize}


\end{frame}


\begin{frame}[fragile]{File and folder paths}
    \begin{itemize}
        \item {\bf working directory:} when using a terminal, you are always working in a
            directory, similar to opening a folder in a file browser
        \item {\bf absolute file paths} start with \verb|/| (Linux, Mac OS) or something like \verb|C:\|
            (Windows), example:
            {\small
            \begin{verbatim}ls /home/peter/lecture-scientific-computing/README.md \end{verbatim}}
        \item {\bf relative paths} are relative to the current working directory of the terminal (with some few unimportant exceptions), example:
            {\small
            \begin{verbatim}ls lecture-scientific-computing/README.md\end{verbatim}}
    \end{itemize}

\end{frame}

\begin{frame}[fragile]{Important command line commands}
    \begin{itemize}
        \item print working directory
\begin{verbatim}$ pwd
/home/peter
\end{verbatim}
        \item change (working) directory
\begin{verbatim}$ cd path/to/folder\end{verbatim}
        \item \verb|ls -l| list directory content
\begin{verbatim}$ ls -l
-rw-rw-r-- 1 peter peter 2,0K Aug  3  2009 book_1.docx
-rw-rw-r-- 1 peter peter 5,0K Aug  3  2009 book_2.docx
\end{verbatim}
    \end{itemize}
\end{frame}


\begin{frame}[fragile]{Helpful tricks}
    \begin{itemize}
        \item abort a running command:
            \begin{verbatim}CTRL + C\end{verbatim}
        \item hitting the TAB key auto-completes your command in many situations, e.g.:
            \begin{verbatim}ls path/to/fol<TAB>\end{verbatim}
        \item previously run commands can be accessed using cursor keys or search:
            \begin{verbatim}CTRL + R\end{verbatim}
        \item run command in background using a trailing \verb|&|, useful for running gitk:
            \begin{verbatim}gitk&\end{verbatim}
    \end{itemize}
\end{frame}


\begin{frame}[fragile]{Getting help}
    Many commands print a help text when called with parameter \verb|-h| or \verb|--help|:
\begin{verbatim}$ sleep --help
Usage: sleep NUMBER[SUFFIX]...
  or:  sleep OPTION
Pause for NUMBER seconds.  SUFFIX may be 's' for
seconds (the default), 'm' for minutes, 'h' for
hours or 'd' for days.
\end{verbatim}

\bigskip
\pause

But googling is fine too!

\bigskip
\pause
Introduction tutorial:\\
\href{https://ubuntu.com/tutorials/command-line-for-beginners}{https://ubuntu.com/tutorials/command-line-for-beginners}

\end{frame}

\end{document}
