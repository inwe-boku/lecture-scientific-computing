\documentclass[10pt,color=usenames,dvipsnames]{beamer}

\usepackage{graphicx}
\usepackage{listings}

\mode<presentation> {

%\usetheme{Madrid}
\usetheme{Boadilla}

\usepackage{qrcode}
\usepackage{multirow}
\usepackage[utf8]{inputenc}

\usepackage{tikz}
\usetikzlibrary{shapes.geometric, arrows}

\usepackage[export]{adjustbox}

\definecolor{bokugreen}{rgb}{0, 0.49, 0}

%\setbeamercolor{palette primary}{fg=bokugreen}
%\setbeamercolor{palette secondary}{bg=white,fg=bokugreen}
%\setbeamercolor{palette tertiary}{bg=white,fg=bokugreen}
%\setbeamercolor{palette quaternary}{bg=white,fg=bokugreen}
\setbeamercolor{structure}{fg=bokugreen} % itemize, enumerate, etc
%\setbeamercolor{section in toc}{fg=bokugreen,bg=white} % TOC sections
\setbeamertemplate{section in toc}[sections numbered]
\setbeamertemplate{subsection in toc}[subsections numbered]


%\usebackgroundtemplate%
%{%
%	\vspace{-0.5cm}
%	\hspace{9.5cm}
%	\includegraphics{bokulogo.png}%
%
%}

\usepackage{hyperref}
\hypersetup{%
    % warning: color makes also QR code colored, therefore commented out
	colorlinks=true,
    allcolors=bokugreen
    % Uargh! Underlining hyperrefs makes troubles with beamer's pause... :-/
    % https://tex.stackexchange.com/q/262651/8964
}

%gets rid of bottom navigation bars
\setbeamertemplate{footline}[frame number]

%gets rid of bottom navigation symbols
\setbeamertemplate{navigation symbols}{}

% Override palette coloring with secondary
%\setbeamercolor{subsection in head/foot}{bg=UBCgrey,fg=white}


%\usecolortheme{lily}
\useoutertheme{infolines}

}

\usepackage{booktabs}
\usepackage{tikz}

% Thin fonts
\usepackage{cmbright}
\usepackage[T1]{fontenc}

% add git commit hash
\usepackage{xstring}
\usepackage{catchfile}
\CatchFileDef{\HEAD}{../.git/refs/heads/master}{}
\newcommand{\gitrevision}{%
  \StrLeft{\HEAD}{7}%
}


%\definecolor{dark_grey}{gray}{0.5}
%\setbeamercolor{normal text}{fg=dark_grey,bg=white}
%\setbeamertemplate{navigation symbols}{}

%\setbeamercolor*{palette primary}{fg=gray!100,bg=gray!10}
%\setbeamercolor*{palette quaternary}{fg=gray!100,bg=gray!10}
%\setbeamercolor*{palette secondary}{fg=gray!100,bg=gray!20}
%\setbeamercolor*{palette tertiary}{fg=gray!100,bg=gray!10}
%\setbeamercolor*{navigation symbols}{fg=white,bg=white}
\usefonttheme{default}

\setbeamertemplate{blocks}[rounded][shadow=false]
%\setbeamercolor{block title}{bg=gray!10}
%\setbeamercolor{block body}{fg=gray,bg=gray!10}
%\setbeamercolor{frametitle}{fg=}

\setbeamertemplate{frametitle}[default][center]

\setbeamertemplate{itemize items}[default]
\setbeamertemplate{enumerate items}[default]

\newcommand{\F}{\mathbb{F}}

\setbeamertemplate{title page}[default][colsep=-4bp,rounded=true]
\setbeamertemplate{frametitle}[default][left]
%\addtobeamertemplate{frametitle}{}{\vspace{4em}} % increase


\title[Scientific Computing]{Scientific Computing}
\author{Peter Regner, Johannes Schmidt}
\institute{Institute for Sustainable Economic Development, BOKU, Wien}

\subtitle{Numpy and the Python Scientific Ecosystem}
\date{2020-04-30}
\begin{document}

% Title Page
\begin{frame}[plain]
    \maketitle
    \begin{center}
        \includegraphics[height=1.7cm]{../common/boku-logo.pdf}\\
    \end{center}
    \vfill
    {
        \tiny
        Latest version available on
        \href{https://github.com/inwe-boku/lecture-scientific-computing/}{Github},
        this PDF is version \gitrevision.
    }
\end{frame}


\begin{frame}
	\tableofcontents
\end{frame}

\section{Misc}
\begin{frame}{Notes}
	\begin{itemize}
		\item Please participate in writing today's lecture notes:
            \href{https://yourpart.eu/p/lecture-scientific-computing04-notes}{https://yourpart.eu/p/lecture-scientific-computing04-notes}
        \item Glossary:\\
            \href{https://yourpart.eu/p/lecture-scientific-computing-glossary}{https://yourpart.eu/p/lecture-scientific-computing-glossary}
	\end{itemize}
\end{frame}

\begin{frame}{Last homework}
    Discussion of homework 03.
    \bigskip
    \pause

	\begin{itemize}
        \item Read and follow the exercise description carefully!
        \item Avoid working on the same notebook at the same time, use git pull/push before
            starting and when your done
	\end{itemize}
\end{frame}

\begin{frame}{Review test}
    See menti.com
\end{frame}


\section{Recap}
\begin{frame}[fragile]{Recap}
    \begin{itemize}[<+->]
        \item how to use command line (\verb|cd|, \verb|ls|, \verb|git|, ...)
        \item git: git pull/push/commit (sorry about all the troubles!)
	    \begin{itemize}
            \item knowing how to use pull/push/commit should suffice for most cases
            \item accidents can be \emph{really} hard to resolve, especially with Jupyter notebooks :(
            \item $\rightarrow$ google error messages, we try to give precise instructions if
                we can foresee difficulties (not always), maybe add an advanced part if time
                permits
	    \end{itemize}
        \item conda, anaconda: install libraries, programming tools and scientific (opensource) software
        \item how to use a Jupyter Notebook (run server, create notebooks, run cells, ...)
        \item basic python syntax (variables, lists, functions, ...)
        \item a quick intro to some data types (integer, float, string, list)
	\end{itemize}
\end{frame}

\begin{frame}[fragile]{Recap: Jupyter notebooks}
    How to run Python code
	\begin{itemize}
        \item Jupyter notebook (this is the only thing we are using in this lecture)
        \item Python terminal
        \item run a Python script from command line
        \item using an IDE/editor to run a script
        \item using an IDE/editor for something fancy (scientific mode with Spyder/PyCharm)
	\end{itemize}

    \pause
    Notebooks are low-threshold and easy to start with, but they can be confusing too.

    \bigskip
    A very good talk about why Jupyter notebooks are bad:
    \href{https://www.youtube.com/watch?v=7jiPeIFXb6U}{https://www.youtube.com/watch?v=7jiPeIFXb6U}\\
    \medskip
    {\small (But you still have to use them, because there is no better alternative at the
    moment.)}
\end{frame}


\section{Numpy and more}
\begin{frame}[fragile]{Numpy and more}
    See lecture04.ipynb!

    \bigskip
    Especially today: ask as many (good) questions as you can!
\end{frame}


\section{Homework assignment}
\begin{frame}[fragile]{Homework assignment}
	\begin{itemize}
		\item Fetch the latest changes from the upstream repository, to get the homework Notebook:
            {\scriptsize
            \begin{verbatim}
cd path/to/homework-scientific-computing
git pull --no-edit upstream master
git push\end{verbatim}
            }
        \item Start Jupyter and solve the exercises in the notebook:
            \href{https://github.com/inwe-boku/homework-scientific-computing/blob/master/homework04-python-scientific-ecosystem//homework04.ipynb}{homework04.ipynb}.
        \item Commit the notebook file and push it to your fork.
	\end{itemize}

    \bigskip
    Do the homework together!

    \bigskip
    Due on 6th of May, 17:00.

    \bigskip
    To avoid merge conflicts, you can either commit a copy of the notebook and add your Github name
    to the filename or solve the exercises together with your group members and \verb|git pull|
    before starting to work on the notebook and \verb|git push| before the next group member starts
    working on it.
\end{frame}

\end{document}
