\documentclass[10pt,color=usenames,dvipsnames]{beamer}

\usepackage{graphicx}
\usepackage{listings}

\mode<presentation> {

%\usetheme{Madrid}
\usetheme{Boadilla}

\usepackage{qrcode}
\usepackage{multirow}
\usepackage[utf8]{inputenc}

\usepackage{tikz}
\usetikzlibrary{shapes.geometric, arrows}

\usepackage[export]{adjustbox}

\definecolor{bokugreen}{rgb}{0, 0.49, 0}

%\setbeamercolor{palette primary}{fg=bokugreen}
%\setbeamercolor{palette secondary}{bg=white,fg=bokugreen}
%\setbeamercolor{palette tertiary}{bg=white,fg=bokugreen}
%\setbeamercolor{palette quaternary}{bg=white,fg=bokugreen}
\setbeamercolor{structure}{fg=bokugreen} % itemize, enumerate, etc
%\setbeamercolor{section in toc}{fg=bokugreen,bg=white} % TOC sections
\setbeamertemplate{section in toc}[sections numbered]
\setbeamertemplate{subsection in toc}[subsections numbered]


%\usebackgroundtemplate%
%{%
%	\vspace{-0.5cm}
%	\hspace{9.5cm}
%	\includegraphics{bokulogo.png}%
%
%}

\usepackage{hyperref}
\hypersetup{%
    % warning: color makes also QR code colored, therefore commented out
	colorlinks=true,
    allcolors=bokugreen
    % Uargh! Underlining hyperrefs makes troubles with beamer's pause... :-/
    % https://tex.stackexchange.com/q/262651/8964
}

%gets rid of bottom navigation bars
\setbeamertemplate{footline}[frame number]

%gets rid of bottom navigation symbols
\setbeamertemplate{navigation symbols}{}

% Override palette coloring with secondary
%\setbeamercolor{subsection in head/foot}{bg=UBCgrey,fg=white}


%\usecolortheme{lily}
\useoutertheme{infolines}

}

\usepackage{booktabs}
\usepackage{tikz}

% Thin fonts
\usepackage{cmbright}
\usepackage[T1]{fontenc}

% add git commit hash
\usepackage{xstring}
\usepackage{catchfile}
\CatchFileDef{\HEAD}{../.git/refs/heads/master}{}
\newcommand{\gitrevision}{%
  \StrLeft{\HEAD}{7}%
}


%\definecolor{dark_grey}{gray}{0.5}
%\setbeamercolor{normal text}{fg=dark_grey,bg=white}
%\setbeamertemplate{navigation symbols}{}

%\setbeamercolor*{palette primary}{fg=gray!100,bg=gray!10}
%\setbeamercolor*{palette quaternary}{fg=gray!100,bg=gray!10}
%\setbeamercolor*{palette secondary}{fg=gray!100,bg=gray!20}
%\setbeamercolor*{palette tertiary}{fg=gray!100,bg=gray!10}
%\setbeamercolor*{navigation symbols}{fg=white,bg=white}
\usefonttheme{default}

\setbeamertemplate{blocks}[rounded][shadow=false]
%\setbeamercolor{block title}{bg=gray!10}
%\setbeamercolor{block body}{fg=gray,bg=gray!10}
%\setbeamercolor{frametitle}{fg=}

\setbeamertemplate{frametitle}[default][center]

\setbeamertemplate{itemize items}[default]
\setbeamertemplate{enumerate items}[default]

\newcommand{\F}{\mathbb{F}}

\setbeamertemplate{title page}[default][colsep=-4bp,rounded=true]
\setbeamertemplate{frametitle}[default][left]
%\addtobeamertemplate{frametitle}{}{\vspace{4em}} % increase


\title[Scientific Computing]{Scientific Computing}
\author{Peter Regner, Johannes Schmidt}
\institute{Institute for Sustainable Economic Development, BOKU, Wien}

\subtitle{Installation of Python in Conda and first steps in Python}
\date{2020-03-26}
\begin{document}

% Title Page
\begin{frame}[plain]
    \maketitle
    \begin{center}
        \includegraphics[height=1.7cm]{../common/boku-logo.pdf}\\
    \end{center}
    \vfill
    {
        \tiny
        Latest version available on
        \href{https://github.com/inwe-boku/lecture-scientific-computing/}{Github},
        this PDF is version \gitrevision.
    }
\end{frame}


\begin{frame}
	\tableofcontents
\end{frame}

\section{Conda}


\begin{frame}{Download...}
	
	Please download Anaconda from
	
	\href{https://www.anaconda.com/distribution/}{https://www.anaconda.com/distribution/}
	
	
\end{frame}

\begin{frame}{Package Manager}
	\begin{itemize}
		
		\item A package (or library in other languages) is a collection of code that helps you accomplish tasks. In this class, when we refer to a package, we refer to Python packages. But be aware that a package, in particular under Linux, may be a complete standalone application.
		\item There are e.g. Python packages for machine learning, for plotting data, for working with tabular data, or for working with matrix-data. More on that later! 
		\item A package manager is there to help you setup a stable environment for your programming system: e.g. it resolves all relevant dependencies when you install a new package.
		\item Conda also allows to create environments: within an environment, you are free to install a different Python verison, different packages and package versions etc. This helps in having a clean seperation between projects. (But also means that you may have installed Python many times on your computer)
	\end{itemize}
	
\end{frame}


\begin{frame}{Package Manager of our choice}
	
	\begin{itemize}
		\item We are going to use conda as Python package manager. 
		\item Anaconda is a system that packages a lot of software together with conda and runs on Windows and Linux. We are going to use it, as it also comes with a graphical user interface, that you may use later. However, we use the command line only. Anaconda provides a lot of software, also for R, you may explore it by starting the graphical user interface. (we do not dig deeper here).
		\item Miniconda is an alternative to use conda. It is smaller and just comes with the core libraries which are necessary to use conda.
		
	\end{itemize}
	
	
\end{frame}

\begin{frame}[fragile]{Anaconda installation}
		
		\begin{itemize}
		\item Run setup program
		\item When asked check the checkbox to set the PATH variable
		\item Open git bash and type
		
		\begin{verbatim}
			conda init bash
		\end{verbatim}
		\item If this works, you successfully installed conda!
		
		\end{itemize}
		
\end{frame}

\begin{frame}[fragile]{Using conda environments}
	
	\begin{itemize}
		\item A conda environment allows you to install a seperate version of Python (or any other conda supported software) with associated packages
		
		\item List all available environments
		\begin{verbatim}
		conda env list
		\end{verbatim}
		
		\item There should be a $base$ environment available on all installations. It is automatically activated when you use conda!
		\item We now first use conda to $update$ - conda!
		\begin{verbatim}
		conda update conda
		\end{verbatim}
		\item This will download and install the newest version of conda
	
	\end{itemize}
	
\end{frame}

\begin{frame}[fragile]{Creating and activating environments}
	
	\begin{itemize}
		\item You can create a new python environment with this command
		\begin{verbatim}
		conda create -n <environment-name> python=3.7 anaconda
		\end{verbatim}
		
		\item \textit{<environment-name>} can be chosen by you. We use \textit{scientific-computing}: 
		\begin{verbatim}
			conda create -n scientific-computing python=3.7 anaconda
		
		\end{verbatim}
		
		\item Again list all available environments
			\begin{verbatim}
			conda env list
			\end{verbatim}
		\item There should be an environment \textit{scientific-computing} available now.
		\item How to work with it? 
		\begin{verbatim}
		conda activate scientific-computing
		\end{verbatim}
		\item Observe how the bash changes - \textit{(scientific-computing)} is shown on the command line now!
		\item Not activating the correct environment is a very common source of problems! Check your environment! 
		
	\end{itemize}
	
\end{frame}

% anaconda vs miniconda
% what is a package manager?
% environments and requirements in the env.yml files

\section{Jupyter}

\begin{frame}{What is a Juypter Notebook?}
	
	\begin{itemize}
		\item In principle, you can write your code in Windows notepad and then run it on the command line
		\item A much better option would be to use an integrated development environment (IDE) such as pyCharm. (Syntax highlighting, code check, etc.)
		\item A different form of coding environment are so called \textit{Jupyter Notebooks}
		\item Here, the code is written on a webpage, sent to a server, executed there, and results are displayed in the notebook again.
		\item Advantages
			\begin{itemize}
				\item Interactive environment for data assessment
				\item Code an be run on server with large computational capacity (if available)
			\end{itemize}
		\item Disadvantages
			\begin{itemize}
				\item Managing large junks of code (such as functions) becomes complicated. 
				\item Using Jupyter notebooks with version control is difficult, , as they introduce their own XML-syntax, which is not really human readable. 
				
			\end{itemize}
		\item Best of two worlds: use an IDE to manage your stable code (functions etc.) and use notebooks for exploration.
		\item For teaching purposes, notebooks are VERY useful. We are going to use them therefore - and may, at some point, introduce additional software such as an IDE.
	\end{itemize}	
	
	
\end{frame}

% how to use jupyter as a calculator
% the dangers of using jupyter

\section{Homework assignment}

\begin{frame}[fragile]{Homework assignment}
	
	\begin{verbatim}
	from cryptography.hazmat.backends import default_backend
	from cryptography.hazmat.primitives import serialization
	
	with open("public_key.pem", "rb") as key_file:
	public_key = serialization.load_pem_public_key(
	key_file.read(),
	backend=default_backend()
	)
	
	
	from cryptography.hazmat.primitives import hashes
	from cryptography.hazmat.primitives.asymmetric import padding
	
	group_members = (
	b"""First Last;1234565;nickname
	First Last;1234565;nickname
	First Last;1234565;nickname
	"""
	)
	
	
	
	encrypted = public_key.encrypt(
	group_members,
	padding.OAEP(
	mgf=padding.MGF1(algorithm=hashes.SHA256()),
	algorithm=hashes.SHA256(),
	label=None
	)
	)
	
	f = open('group_members.txt', 'wb')
	f.write(encrypted)
	f.close()
	\end{verbatim}
	
\end{frame}

\end{document}
