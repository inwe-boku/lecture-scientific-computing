\documentclass[10pt,color=usenames,dvipsnames]{beamer}

\usepackage{graphicx}
\usepackage{listings}

\mode<presentation> {

%\usetheme{Madrid}
\usetheme{Boadilla}

\usepackage{qrcode}
\usepackage{multirow}
\usepackage[utf8]{inputenc}

\usepackage{tikz}
\usetikzlibrary{shapes.geometric, arrows}

\usepackage[export]{adjustbox}

\definecolor{bokugreen}{rgb}{0, 0.49, 0}

%\setbeamercolor{palette primary}{fg=bokugreen}
%\setbeamercolor{palette secondary}{bg=white,fg=bokugreen}
%\setbeamercolor{palette tertiary}{bg=white,fg=bokugreen}
%\setbeamercolor{palette quaternary}{bg=white,fg=bokugreen}
\setbeamercolor{structure}{fg=bokugreen} % itemize, enumerate, etc
%\setbeamercolor{section in toc}{fg=bokugreen,bg=white} % TOC sections
\setbeamertemplate{section in toc}[sections numbered]
\setbeamertemplate{subsection in toc}[subsections numbered]


%\usebackgroundtemplate%
%{%
%	\vspace{-0.5cm}
%	\hspace{9.5cm}
%	\includegraphics{bokulogo.png}%
%
%}

\usepackage{hyperref}
\hypersetup{%
    % warning: color makes also QR code colored, therefore commented out
	colorlinks=true,
    allcolors=bokugreen
    % Uargh! Underlining hyperrefs makes troubles with beamer's pause... :-/
    % https://tex.stackexchange.com/q/262651/8964
}

%gets rid of bottom navigation bars
\setbeamertemplate{footline}[frame number]

%gets rid of bottom navigation symbols
\setbeamertemplate{navigation symbols}{}

% Override palette coloring with secondary
%\setbeamercolor{subsection in head/foot}{bg=UBCgrey,fg=white}


%\usecolortheme{lily}
\useoutertheme{infolines}

}

\usepackage{booktabs}
\usepackage{tikz}

% Thin fonts
\usepackage{cmbright}
\usepackage[T1]{fontenc}

% add git commit hash
\usepackage{xstring}
\usepackage{catchfile}
\CatchFileDef{\HEAD}{../.git/refs/heads/master}{}
\newcommand{\gitrevision}{%
  \StrLeft{\HEAD}{7}%
}


%\definecolor{dark_grey}{gray}{0.5}
%\setbeamercolor{normal text}{fg=dark_grey,bg=white}
%\setbeamertemplate{navigation symbols}{}

%\setbeamercolor*{palette primary}{fg=gray!100,bg=gray!10}
%\setbeamercolor*{palette quaternary}{fg=gray!100,bg=gray!10}
%\setbeamercolor*{palette secondary}{fg=gray!100,bg=gray!20}
%\setbeamercolor*{palette tertiary}{fg=gray!100,bg=gray!10}
%\setbeamercolor*{navigation symbols}{fg=white,bg=white}
\usefonttheme{default}

\setbeamertemplate{blocks}[rounded][shadow=false]
%\setbeamercolor{block title}{bg=gray!10}
%\setbeamercolor{block body}{fg=gray,bg=gray!10}
%\setbeamercolor{frametitle}{fg=}

\setbeamertemplate{frametitle}[default][center]

\setbeamertemplate{itemize items}[default]
\setbeamertemplate{enumerate items}[default]

\newcommand{\F}{\mathbb{F}}

\setbeamertemplate{title page}[default][colsep=-4bp,rounded=true]
\setbeamertemplate{frametitle}[default][left]
%\addtobeamertemplate{frametitle}{}{\vspace{4em}} % increase


\title[Scientific Computing]{Scientific Computing}
\author{Peter Regner, Johannes Schmidt}
\institute{Institute for Sustainable Economic Development, BOKU, Wien}

\subtitle{Start programming in Python}
\date{2020-04-02}
\begin{document}

% Title Page
\begin{frame}[plain]
    \maketitle
    \begin{center}
        \includegraphics[height=1.7cm]{../common/boku-logo.pdf}\\
    \end{center}
    \vfill
    {
        \tiny
        Latest version available on
        \href{https://github.com/inwe-boku/lecture-scientific-computing/}{Github},
        this PDF is version \gitrevision.
    }
\end{frame}


\begin{frame}
	\tableofcontents
\end{frame}

\section{Intro}

% variables
% functions
% if, for (while)

\section{Leftover from last week}

\begin{frame}[fragile]{Using Jupyter Notebooks}
	
	If you want to use conda environments in your notebooks, do the following:
	\begin{itemize}
		\item Install \textit{ipykernel} in your environment
		
		\begin{verbatim}
		conda activate <conda-environment>
		conda install ipykernel
		conda deactivate
		\end{verbatim}
		\item Additionally, install \textit{nb\_conda\_kernels} in \textit{base} if you run your jupyter notebook from base:
		\begin{verbatim}
		conda activate base    # could be also some other environment
		conda install nb_conda_kernels
		\end{verbatim}
	\end{itemize}
\end{frame}

\begin{frame}[fragile]{A hint on replacements in exercises and <>}
	
	If you see text enclosed in <> on our slides or in the notebook such as
	\begin{verbatim}
	<something here>
	\end{verbatim}
	it means that 'something here' should be replaced by some input from your side. (Hopefully obvious from the context)
	
	Anything else can just be copied.

\end{frame}


\section{Homework assignment}


\begin{frame}[fragile]{Homework assignment}

    Write a function, which calculates the number of new infections at time $t$.

    \[
        \frac{e^{-(x-\mu)/s}} {s\left(1+e^{-(x-\mu)/s}\right)^2}
    \]

    Then use the following code to plot it:
    % TODO would be nice to prepare a jupyter notebook with widgets to adjust parameters!
\end{frame}

\end{document}
