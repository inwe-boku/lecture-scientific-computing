\documentclass[10pt,color=usenames,dvipsnames]{beamer}

\usepackage{graphicx}
\usepackage{listings}

\mode<presentation> {

%\usetheme{Madrid}
\usetheme{Boadilla}

\usepackage{qrcode}
\usepackage{multirow}
\usepackage[utf8]{inputenc}

\usepackage{tikz}
\usetikzlibrary{shapes.geometric, arrows}

\usepackage[export]{adjustbox}

\definecolor{bokugreen}{rgb}{0, 0.49, 0}

%\setbeamercolor{palette primary}{fg=bokugreen}
%\setbeamercolor{palette secondary}{bg=white,fg=bokugreen}
%\setbeamercolor{palette tertiary}{bg=white,fg=bokugreen}
%\setbeamercolor{palette quaternary}{bg=white,fg=bokugreen}
\setbeamercolor{structure}{fg=bokugreen} % itemize, enumerate, etc
%\setbeamercolor{section in toc}{fg=bokugreen,bg=white} % TOC sections
\setbeamertemplate{section in toc}[sections numbered]
\setbeamertemplate{subsection in toc}[subsections numbered]


%\usebackgroundtemplate%
%{%
%	\vspace{-0.5cm}
%	\hspace{9.5cm}
%	\includegraphics{bokulogo.png}%
%
%}

\usepackage{hyperref}
\hypersetup{%
    % warning: color makes also QR code colored, therefore commented out
	%colorlinks=true
	% hyperlinks will be colored
	%allcolors=ERC7
}



%gets rid of bottom navigation bars
\setbeamertemplate{footline}[frame number]

%gets rid of bottom navigation symbols
\setbeamertemplate{navigation symbols}{}

% Override palette coloring with secondary
%\setbeamercolor{subsection in head/foot}{bg=UBCgrey,fg=white}


%\usecolortheme{lily}
\useoutertheme{infolines}

}


\usepackage{booktabs} 
\usepackage{tikz}


% Thin fonts
\usepackage{cmbright}
\usepackage[T1]{fontenc}

%\definecolor{dark_grey}{gray}{0.5}
%\setbeamercolor{normal text}{fg=dark_grey,bg=white}
%\setbeamertemplate{navigation symbols}{}

%\setbeamercolor*{palette primary}{fg=gray!100,bg=gray!10}
%\setbeamercolor*{palette quaternary}{fg=gray!100,bg=gray!10}
%\setbeamercolor*{palette secondary}{fg=gray!100,bg=gray!20}
%\setbeamercolor*{palette tertiary}{fg=gray!100,bg=gray!10}
%\setbeamercolor*{navigation symbols}{fg=white,bg=white}
\usefonttheme{default}


\setbeamertemplate{blocks}[rounded][shadow=false]
%\setbeamercolor{block title}{bg=gray!10}
%\setbeamercolor{block body}{fg=gray,bg=gray!10}
%\setbeamercolor{frametitle}{fg=}

\setbeamertemplate{frametitle}[default][center]

\setbeamertemplate{itemize items}[default]
\setbeamertemplate{enumerate items}[default]

\newcommand{\F}{\mathbb{F}}

\setbeamertemplate{title page}[default][colsep=-4bp,rounded=true]
\setbeamertemplate{frametitle}[default][left]
%\addtobeamertemplate{frametitle}{}{\vspace{4em}} % increase


\title[Scientific Computing]{Scientific Computing}
\subtitle{Introduction}

\author{Peter Regner, Johannes Schmidt}
\institute{Institute for Sustainable Economic Development, BOKU, Wien}
\date{12.3.2020}

\begin{document}

% Title Page
\begin{frame}[plain]
    \maketitle
    \begin{center}
        \includegraphics[height=1.7cm]{boku-logo.pdf}\\
    \end{center}
\end{frame}

\begin{frame}{Survey regarding preknowledge}
    \begin{center}
        \textcolor{black}{
            \qrcode[height=4.5cm]{https://www.menti.com/<snip>}
        }\\
        \vspace{0.5cm}
        \huge{
            or\\
            menti.com\\
            code <snip>
        }
    \end{center}
\end{frame}

\begin{frame}
	
	\tableofcontents
	
\end{frame}

\section{Who we are}

\begin{frame}{Who we are}
    Johannes Schmidt <johannes.schmidt@boku.ac.at>\\
    \begin{itemize}
        \item
    \end{itemize}

    \pause
    \bigskip

    Peter Regner <peter.regner@boku.ac.at>
    \begin{itemize}
        \item PhD student at the Institute for Sustainable Economic Development
        \item worked almost 7 years as Python developer in semiconductor industry
        \item studied mathematics at the TU Wien
    \end{itemize}
\end{frame}

\section{What this course is about}

\begin{frame}{Aim of course}
	
	Learn to use programming as a tool in research
	
	\begin{itemize}
		\item Version control your code with git and collaborate on git(hub)
		\item Install python, python packages, and notebook servers in conda environments
		\item Understand composition of python scientific stack
		\item Start programming in python and understand flow featurs (loops and conditions), functions, classes and objects
		\item Use python stack packages (numpy, xarray)
		\item Use open (climate) data in your research projects
		\item Generate plots
	\end{itemize}
	
\end{frame}

\section{How the class works}

\begin{frame}{How the class works}
	
	\begin{itemize}
		\item 1.5 hours of lecture
		\item 1.5 hours of practical exercises
		\item Homework in groups
	\end{itemize}	

\end{frame}

\section{Prerequisites}

\begin{frame}{Prerequisites}
	
	\begin{itemize}
		\item You should feel very comfortable with your respective operating system (Windows, Linux, Mac OS)
		\item Ideally, you have some experience with using a command line / terminal
		\item In a perfect world, you have programmed before
	\end{itemize}
	
\end{frame}


\section{How we grade}

\begin{frame}{Grading scheme}

	\begin{itemize}
		\item Organize in groups of 4 students	
		\item Do homework together
		\item Be prepared to present homework individually ("Kreuzerlübung")		
	\end{itemize}

\end{frame}

\section{Resources}

\begin{frame}{Resources}
	
	\begin{itemize}
		\item Our forum on... Post there! Do NOT write emails to us.
		\item Your fellow peers in your group!
		\item Us, in class
	\end{itemize}


\end{frame}

\section{First assignment}

\begin{frame}{First assignment (I)}


Homework (due on 19th of March):

\begin{itemize}
	
	\item register at github.com or any alternative GIT hoster (e.g. gitlab.com)
	\item install a GIT and gitk (see instructions (1) below)
	\item click/browse quickly through links \& tutorials (see links below)
	\item  clone a repository, change a file and commit (see instructions (2) below)
	\item prepare one question about GIT and add it to the list: \href{https://yourpart.eu/p/lecture-scientific-computing}{https://yourpart.eu/p/lecture-scientific-computing}

\end{itemize}


\end{frame}

\begin{frame}{First assignment (II)}
	(1) Install GIT \& gitk
	
	Linux: on Debian based systems run the command:
	
	sudo apt install git gitk
	
		Windows:
	(a) install notepad++ (https://notepad-plus-plus.org/downloads/)
	(b) install git for windows (https://gitforwindows.org/). During the installation procedure, choose notepad++ as your preferred edtior.
	
	(2) Clone the lecture repository
	
	Run a terminal (windows: GIT bash) and run the command:
	
	%\begin{verbatim}
	git clone https://github.com/inwe-boku/lecture-scientific-computing.git
	%\end{verbatim}
	
	Then change a file of your choice and run:
	
	%\begin{verbatim}
	git add <file-name-of-the-file-you-changed>
	%\end{verbatim}
	
	Explain your change in the commit message and close the editor.
	
	Run gitk to check your commit.
	
	See also:
	\href{https://guides.github.com/introduction/git-handbook/}{https://guides.github.com/introduction/git-handbook/}
	(especially "Example: Contribute to an existing repository")
	
	
	More helpful resources about GIT can be found here:
	\href{https://github.com/inwe-boku/lecture-scientific-computing/blob/master/lecture01-git-version-control/links.rst}
	{https://github.com/inwe-boku/lecture-scientific-computing/blob/master/lecture01-git-version-control/links.rst}

\end{frame}



\end{document} 
